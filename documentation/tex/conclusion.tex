This template and the writing guidelines should help achieving a consistently formatted and clear documents. A similar template is also available for Word.

Every writing and presentation must have a conclusion. This fact is here emphasized by having this short and rather artificial summary also in this template. A concise summary table can be a good way for providing an overview of the most important points.

Lastly, some final points regarding this template. To prepare the document, use the pdf\LaTeX{} compiler. This option is quite easily found in most \LaTeX{} editors, and when using the terminal simply choose \texttt{pdflatex} as the command. The (list of) references is created using the biber program, found similarly. To typeset the list of abbreviations and symbols requires a run of makeindex on the document. If it seems like the table of contents or the cross-references do not display correctly, try compiling the document again using pdf\LaTeX{}. Finally, if you run into errors while compiling, make sure that your \TeX{} distribution is up to date.

The template has been developed in the Overleaf environment, and the author warmly recommends the usage of its version 2 in writing theses. The easiest way to gain access to the document template is to ask for a copy link to the project from the supervisor of the work, or from the maintainer of this template. To use Overleaf, however, requires a user account and constant access to the Internet.

Hopefully, an up-to-date version of the template is also available in the university intranet. The template has been tested and found to work in the Windows system Mik\TeX{} and the Unix-based system \TeX{} Live environments. The first of these can automatically install the possibly missing packages, but with the latter you may have to install them, or up-to-date versions of them manually.